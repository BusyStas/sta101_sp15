% -*- TeX-engine: xetex; eval: (auto-fill-mode 0); eval: (visual-line-mode 1); -*-
% Compile with XeLaTeX

%%%%%%%%%%%%%%%%%%%%%%%
% To do before class
%%%%%%%%%%%%%%%%%%%%%%%

% Send the Logistics/Week0Annoucnement (the night before).
% Send an email reminding students to bring a charged computer (the night before).

%%%%%%%%%%%%%%%%%%%%%%%
% Option 1: Slides: (comment for handouts)   %
%%%%%%%%%%%%%%%%%%%%%%%

%\documentclass[slidestop,compress,mathserif,12pt,t,professionalfonts,xcolor=table]{beamer}
%
%% solution stuff
%\newcommand{\solnMult}[1]{
%\only<1>{#1}
%\only<2->{\red{\textbf{#1}}}
%}
%\newcommand{\soln}[1]{\textit{#1}}
%
%\newcommand{\solnMultOn}[3]{
%\only<#1>{#3}
%\only<{#2}->{\red{\textbf{#3}}}
%}

%%%%%%%%%%%%%%%%%%%%%%%%%%%%%%%
% Option 2: Handouts, without solutions (post before class)    %
%%%%%%%%%%%%%%%%%%%%%%%%%%%%%%%

 \documentclass[11pt,containsverbatim,handout,xcolor=xelatex,dvipsnames,table]{beamer}

 % handout layout
 \usepackage{pgfpages}
 \pgfpagesuselayout{4 on 1}[letterpaper,landscape,border shrink=5mm]

 % solution stuff
 \newcommand{\solnMult}[1]{#1}
 \newcommand{\soln}[1]{}
 \newcommand{\solnMultOn}[3]{#3}

 % % This breaks things for me for some reason.
 % tell pgfpages how to set page sizes in XeLaTeX
 \renewcommand\pgfsetupphysicalpagesizes{%
    \pdfpagewidth\pgfphysicalwidth\pdfpageheight\pgfphysicalheight%
 }

%%%%%%%%%%%%%%%%%%%%%%%%%%%%%%%%%%%%
% Option 3: Handouts, with solutions (may post after class if need be)    %
%%%%%%%%%%%%%%%%%%%%%%%%%%%%%%%%%%%%

% \documentclass[11pt,containsverbatim,handout,xcolor=xelatex,dvipsnames,table]{beamer}

% % handout layout
% \usepackage{pgfpages}
% \pgfpagesuselayout{4 on 1}[letterpaper,landscape,border shrink=5mm]

% % solution stuff
% \newcommand{\solnMult}[1]{\red{\textbf{#1}}}
% \newcommand{\soln}[1]{\textit{#1}}

% % % This breaks things for me for some reason.
% % % tell pgfpages how to set page sizes in XeLaTeX
% % \renewcommand\pgfsetupphysicalpagesizes{%
% %    \pdfpagewidth\pgfphysicalwidth\pdfpageheight\pgfphysicalheight%
% % }

%%%%%%%%%%%%%%%%%%%%%%%%%%%%%%%
% Option 4: Notes Only
%%%%%%%%%%%%%%%%%%%%%%%%%%%%%%%

% % See http://tex.stackexchange.com/questions/114219/add-notes-to-latex-beamer
% \documentclass[10pt,containsverbatim,xcolor=xelatex,dvipsnames,table,notes=only]{beamer}

% % handout layout
% % \usepackage{pgfpages}
% % \pgfpagesuselayout{1 on 1}[letterpaper, landscape, border shrink=5mm]

% % solution stuff
% \newcommand{\solnMult}[1]{#1}
% \newcommand{\soln}[1]{}

% % % Having a problem with this.
% % tell pgfpages how to set page sizes in XeLaTeX
% % \renewcommand\pgfsetupphysicalpagesizes{%
% %   \pdfpagewidth\pgfphysicalwidth\pdfpageheight\pgfphysicalheight%
% %}

%%%%%%%%%%
% Load style file, defaults  %
%%%%%%%%%%

%%%%%%%%%%%%%%%%
% Themes
%%%%%%%%%%%%%%%%

% See http://deic.uab.es/~iblanes/beamer_gallery/ for mor options

% Style theme
\usetheme{Pittsburgh}

% Color theme
\usecolortheme{seahorse}

% Helvetica Neue Light for most text
\usepackage{fontspec}
\setsansfont{Helvetica Neue Light}

%%%%%%%%%%%%%%%%
% Packages
%%%%%%%%%%%%%%%%

\usepackage{geometry}
\usepackage{graphicx}
\usepackage{amssymb}
\usepackage{epstopdf}
\usepackage{amsmath}  	% this permits text in eqnarray among other benefits
\usepackage{url}		% produces hyperlinks
\usepackage[english]{babel}
\usepackage{colortbl}	% allows for color usage in tables
\usepackage{multirow}	% allows for rows that span multiple rows in tables
\usepackage{color}		% this package has a variety of color options
\usepackage{pgf}
\usepackage{calc}
\usepackage{ulem}
\usepackage{multicol}
\usepackage{textcomp}
\usepackage{listings}
\usepackage{changepage}
\usepackage{tikz}
\usetikzlibrary{trees}		% for probability trees
\usepackage{fancyvrb}	% for colored code chunks
\usepackage{nameref}

%%%%%%%%%%%%%%%%
% Remove navigation symbols
%%%%%%%%%%%%%%%%

\beamertemplatenavigationsymbolsempty
\hypersetup{pdfpagemode=UseNone} % don't show bookmarks on initial view

%%%%%%%%%%%%%%%%
% User defined colors
%%%%%%%%%%%%%%%%

% Pantone 2015 Spring colors
% http://iwork3.us/2014/09/16/pantone-2015-spring-fashion-report/
% update each semester or year

\xdefinecolor{custom_blue}{rgb}{0, 0.70, 0.79} % scuba blue
\xdefinecolor{custom_darkBlue}{rgb}{0.11, 0.31, 0.54} % classic blue
\xdefinecolor{custom_orange}{rgb}{0.97, 0.57, 0.34} % tangerine
\xdefinecolor{custom_green}{rgb}{0.49, 0.81, 0.71} % lucite green
\xdefinecolor{custom_red}{rgb}{0.58, 0.32, 0.32} % marsala

\xdefinecolor{custom_lightGray}{rgb}{0.78, 0.80, 0.80} % glacier gray
\xdefinecolor{custom_darkGray}{rgb}{0.54, 0.52, 0.53} % titanium

%%%%%%%%%%%%%%%%
% Template colors
%%%%%%%%%%%%%%%%

\setbeamercolor*{palette primary}{fg=white,bg= custom_blue}
\setbeamercolor*{palette secondary}{fg=black,bg= custom_blue!80!black}
\setbeamercolor*{palette tertiary}{fg=white,bg= custom_blue!80!black!80}
\setbeamercolor*{palette quaternary}{fg=white,bg= custom_blue}

\setbeamercolor{structure}{fg= custom_blue}
\setbeamercolor{frametitle}{bg= custom_blue!90}
\setbeamertemplate{blocks}[shadow=false]
\setbeamersize{text margin left=2em,text margin right=2em}

%%%%%%%%%%%%%%%%
% Styling fonts, bullets, etc.
%%%%%%%%%%%%%%%%

% title slide
\setbeamerfont{title}{size=\large,series=\bfseries}
\setbeamerfont{subtitle}{size=\large,series=\mdseries}
%\setbeamerfont{institute}{size=\large,series=\mdseries}

% color of alerted text
\setbeamercolor{alerted text}{fg=custom_orange}

% styling of itemize bullets
\setbeamercolor{item}{fg=custom_blue}
\setbeamertemplate{itemize item}{{{\small$\blacktriangleright$}}}
\setbeamercolor{subitem}{fg=custom_blue}
\setbeamertemplate{itemize subitem}{{\textendash}}
\setbeamerfont{itemize/enumerate subbody}{size=\footnotesize}
\setbeamerfont{itemize/enumerate subitem}{size=\footnotesize}

% styling of enumerate bullets
\setbeamertemplate{enumerate item}{\insertenumlabel.}
\setbeamerfont{enumerate item}{family={\fontspec{Helvetica Neue}}}
\setbeamerfont{enumerate subitem}{family={\fontspec{Helvetica Neue}}}
\setbeamerfont{enumerate subsubitem}{family={\fontspec{Helvetica Neue}}}

% make frame titles small to make room in the slide
\setbeamerfont{frametitle}{size=\small} 

% set Helvetica Neue font for frame and section titles
\setbeamerfont{frametitle}{family={\fontspec{Helvetica Neue}}}
\setbeamerfont{sectiontitle}{family={\fontspec{Helvetica Neue}}}
\setbeamerfont{section in toc}{family={\fontspec{Helvetica Neue}}}
\setbeamerfont{subsection in toc}{family={\fontspec{Helvetica Neue}}, size=\small}
\setbeamerfont{footline}{family={\fontspec{Helvetica Neue}}}
\setbeamerfont{subsection in toc}{family={\fontspec{Helvetica Neue}}}
\setbeamerfont{block title}{family={\fontspec{Helvetica Neue}}}

%%%%%%%%%%%%%%%%
% New fonts accessed by fontspec package
%%%%%%%%%%%%%%%%

% Monaco font for code
\newfontfamily{\monaco}{Monaco}

%%%%%%%%%%%%%%%%
% Color text commands
%%%%%%%%%%%%%%%%

%orange
\newcommand{\orange}[1]{\textit{\textcolor{custom_orange}{#1}}}

% green
\newcommand{\green}[1]{\textit{\textcolor{custom_green}{#1}}}

% red
\newcommand{\red}[1]{\textit{\textcolor{custom_red}{#1}}}

% dark gray
\newcommand{\darkgray}[1]{\textit{\textcolor{custom_darkGray}{#1}}}

% light gray
\newcommand{\lightgray}[1]{\textit{\textcolor{custom_lightGray}{#1}}}

% pink
\newcommand{\pink}[1]{\textit{\textcolor{pink}{#1}}}


%%%%%%%%%%%%%%%%
% Custom commands
%%%%%%%%%%%%%%%%

% empty box for probability tree frame
\newcommand{\emptybox}[2]{
	\fbox{ \begin{minipage}{#1} \hfill\vspace{#2} \end{minipage} }
}

% cancel
\newcommand{\cancel}[1]{%
    \tikz[baseline=(tocancel.base)]{
        \node[inner sep=0pt,outer sep=0pt] (tocancel) {#1};
        \draw[red, line width=0.5mm] (tocancel.south west) -- (tocancel.north east);
    }%
}

% degree
\newcommand{\degree}{\ensuremath{^\circ}}

% cite
\newcommand{\ct}[1]{
\vfill
{\tiny #1}}

% Note
\newcommand{\Note}[1]{
\rule{2.5cm}{0.25pt} \\ \textit{\footnotesize{\textcolor{custom_red}{Note:} \textcolor{custom_darkGray}{#1}}}}

% Remember
\newcommand{\Remember}[1]{\textit{\scriptsize{\textcolor{custom_red}{Remember:} #1}}}

% links: webURL, webLink
\newcommand{\webURL}[1]{\urlstyle{same}{\textit{\textcolor{custom_blue}{\url{#1}}}}}
\newcommand{\webLink}[2]{\href{#1}{\textcolor{custom_blue}{{#2}}}}

% mail
\newcommand{\mail}[1]{\href{mailto:#1}{\textit{\textcolor{custom_blue}{#1}}}}

% highlighting: hl, hlGr, mathhl
\newcommand{\hl}[1]{\textit{\textcolor{custom_blue}{#1}}}
\newcommand{\hlGr}[1]{\textit{\textcolor{custom_green}{#1}}}
\newcommand{\mathhl}[1]{\textcolor{custom_blue}{\ensuremath{#1}}}

% example
\newcommand{\ex}[1]{\textcolor{blue}{{{\small (#1)}}}}

% two col: two columns
\newenvironment{twocol}[4]{
\begin{columns}[c]
\column{#1\textwidth}
#3
\column{#2\textwidth}
#4
\end{columns}
}

% slot (for probability calculations)
\newenvironment{slot}[2]{
\begin{array}{c} 
\underline{#1} \\ 
#2
\end{array}
}

% pr: left and right parentheses
\newcommand{\pr}[1]{
\left( #1 \right)
}

%%%%%%%%%%%%%%%%
% Custom blocks
%%%%%%%%%%%%%%%%

% activity: less commonly used
\newcommand{\activity}[2]{
\setbeamertemplate{itemize item}{{{\small\textcolor{custom_orange}{$\blacktriangleright$}}}}
\setbeamercolor{block title}{fg=white, bg=custom_orange}
\setbeamerfont{block title}{size=\small}
\setbeamercolor{block body}{fg=black, bg=custom_orange!20!white!80}
\setbeamerfont{block body}{size=\small}
\begin{block}{Activity: #1}
\setlength\abovedisplayskip{0pt}
#2
\end{block}
}

% app: application exercise
\newcommand{\app}[2]{
\setbeamercolor{block title}{fg=white,bg=custom_green}
\setbeamercolor{block body}{fg=black,bg=custom_green!20!white!80}
\begin{block}{{\small Application exercise: #1}}
#2
\end{block}
}

% disc: discussion question
\newcommand{\disc}[1]{
\vspace*{-2ex}
\setbeamercolor{block body}{bg=custom_blue!25!white!80, fg=custom_blue!55!black!95}
\begin{block}{\vspace*{-3ex}}
#1
\end{block}
\vspace*{-1ex}
}

% clicker: clicker question
\newcommand{\clicker}[1]{
\setbeamercolor{block title}{bg=custom_blue!80!white!50,fg=custom_blue!30!black!90}
\setbeamercolor{block body}{bg=custom_blue!20!white!80,fg=custom_blue!30!black!90}
\begin{block}{\vspace*{-0.2ex}{\footnotesize Clicker question}\vspace*{-0.2ex}}
#1
\end{block}
}

% formula
\newcommand{\formula}[2]{
\setbeamercolor{block title}{bg=custom_blue!40!white!60,fg=custom_blue!55!black!95}
\begin{block}{{\small#1}}
#2
\end{block}
}

% code
\newcommand{\Rcode}[1]{
{\monaco {\footnotesize \textcolor{custom_darkBlue}{#1}}}
}

% output
\newcommand{\Rout}[1]{
{\monaco {\footnotesize \textcolor{custom_darkGray}{#1}}}
}

%%%%%%%%%%%%%%%%
% Change margin
%%%%%%%%%%%%%%%%

\newenvironment{changemargin}[2]{%
\begin{list}{}{%
\setlength{\topsep}{0pt}%
\setlength{\leftmargin}{#1}%
\setlength{\rightmargin}{#2}%
\setlength{\listparindent}{\parindent}%
\setlength{\itemindent}{\parindent}%
\setlength{\parsep}{\parskip}%
}%
\item}{\end{list}}

%%%%%%%%%%%%%%%%
% Footnote
%%%%%%%%%%%%%%%%

\long\def\symbolfootnote[#1]#2{\begingroup%
\def\thefootnote{\fnsymbol{footnote}}\footnote[#1]{#2}\endgroup}

%%%%%%%%%%%%%%%%
% Graphics
%%%%%%%%%%%%%%%%

\DeclareGraphicsRule{.tif}{png}{.png}{`convert #1 `dirname #1`/`basename #1 .tif`.png}

%%%%%%%%%%%%%%%%
% Slide number
%%%%%%%%%%%%%%%%

\setbeamertemplate{footline}{%
    \raisebox{5pt}{\makebox[\paperwidth]{\hfill\makebox[20pt]{\color{gray}
          \scriptsize\insertframenumber}}}\hspace*{5pt}}

          
%%%%%%%%%%%%%%%%
% Remove page numbers
%%%%%%%%%%%%%%%%

\newcommand{\removepagenumbers}{% 
  \setbeamertemplate{footline}{}
}

%%%%%%%%%%%%%%%%
% TOC slides
%%%%%%%%%%%%%%%%

\setbeamertemplate{section in toc}{\inserttocsectionnumber.~\inserttocsection}
\setbeamertemplate{subsection in toc}{$\qquad$\inserttocsubsectionnumber.~\inserttocsubsection \\}

\AtBeginSection[] 
{ 
  \addtocounter{framenumber}{-1} 
  % 
  {\removepagenumbers 
  {\small
    \begin{frame}<beamer> 
    \frametitle{Outline} 
    \tableofcontents[currentsection] 
  \end{frame} 
  } 
  }
} 

\AtBeginSubsection[] 
{ 
  \addtocounter{framenumber}{-1} 
  % 
  {\removepagenumbers 
  {\small
    \begin{frame}<beamer> 
    \frametitle{Outline} 
    \tableofcontents[currentsection,currentsubsection] 
  \end{frame} 
  } 
  }
}
\input{../../definitions_default.tex}
% ALT ALT
%\input{../../definitions_custom.tex}

%%%%%%%%%%%
% Cover slide info    %
%%%%%%%%%%%

\title{Review}
\subtitle{Bayesian vs. frequentist inference}
\author{Sta 101 - Spring 2015}
\date{April 20, 2015}
\institute{Duke University, Department of Statistical Science}

%%%%%%%%%%%
% Begin document   %
%%%%%%%%%%%

\begin{document}

%%%%%%%%%%%%%%%%%%%%%%%%%%%%%%%%%%%%

% Title Page

\begin{frame}[plain]

\titlepage
\vfill
{\scriptsize \webLink{\PersonalSite}{Dr. \LastName{}} \hfill Slides posted at  \webLink{\CourseSite}{\CourseSite}}
\addtocounter{framenumber}{-1} 

\end{frame}

%%%%%%%%%%%%%%%%%%%%%%%%%%%%%%%%%%%%

\section{Housekeeping}

%%%%%%%%%%%%%%%%%%%%%%%%%%%%%%%%%%%%

\begin{frame}
\frametitle{Announcements}

\begin{itemize}

\item Poster sessions in Link Classroom 3 tomorrow, submission on Sakai due by your lab session time.

\item Posttest due by Friday 5/1 at midnight

\item Final exam review: Thursday, 4/30, 5:30 - 6:30pm

\end{itemize}

\end{frame}

%%%%%%%%%%%%%%%%%%%%%%%%%%%%%%%%%%%%

\section{Bayesian vs. frequentist Inference}

%%%%%%%%%%%%%%%%%%%%%%%%%%%%%%%%%%%%

\begin{frame}
\frametitle{M\&Ms}

\begin{itemize}

\item We have a population of M\&Ms. The percentage of yellow M\&Ms is either 10\% or 20\%.

\item You have been hired as a statistical consultant to decide whether the true percentage of yellow M\&Ms is 10\%. You are being asked to make a decision, and there are associated payoff/losses that you should consider.

\end{itemize}

\end{frame}

%%%%%%%%%%%%%%%%%%%%%%%%%%%%%%%%%%%%

\begin{frame}
\frametitle{Decision table}

\begin{center}
\renewcommand\arraystretch{1.5}
\begin{tabular}{l | p{3cm} | p{3cm}}
				&   \multicolumn{2}{c}{True state of the population} \\
\hline
Decision			& \% yellow = 10\%		& \% yellow = 20\% \\
\hline
\% yellow = 10\%	& \green{Your boss gives you a bonus, and I'll bring you candy on Wednesday}	& \red{You lose your job, and no candy for you} \\
\hline
\%yellow = 20\%	& \red{You lose your job, and no candy for you} 	& \green{Your boss gives you a bonus, and I'll bring you candy on Wednesday} \\
\end{tabular}
\end{center}

\end{frame}

%%%%%%%%%%%%%%%%%%%%%%%%%%%%%%%%%%%%

\begin{frame}
\frametitle{Data}

\begin{itemize}

\item I will show you a random sample from the population, but you pay \$200 for each M\&M, and you must buy in \$1000 increments.  

\item That is, you may buy 5, 10, 15, or 20 M\&Ms.

\end{itemize}

\end{frame}

%%%%%%%%%%%%%%%%%%%%%%%%%%%%%%%%%%%%

\subsection{Frequentist inference}

%%%%%%%%%%%%%%%%%%%%%%%%%%%%%%%%%%%%

\begin{frame}
\frametitle{Frequentist inference}

\begin{itemize}

\item Hypotheses:
\begin{itemize}
\item $H_0$: 10\% yellow M\&Ms
\item $H_A$: more than 10\% yellow M\&Ms 
\end{itemize}

\item Your test statistic is the number of yellow M\&Ms you observe in the sample. 

\item The p-value will be the probability of observing this many or more yellow M\&Ms given the null hypothesis is true.

\end{itemize}

\end{frame}

%%%%%%%%%%%%%%%%%%%%%%%%%%%%%%%%%%%%

\begin{frame}
\frametitle{Setup}

\app{Set up -- data: clicker}{[CLICKER] How many M\&Ms would you buy? Decide as a team and vote.}

\begin{multicols}{4}
\begin{enumerate}[(a)]
\item 5
\item 10
\item 15
\item 20
\end{enumerate}
\end{multicols}

\pause

\app{Set up -- significance level}{[CLICKER] Discuss at what significance level you will reject the null hypothesis. Submit a value between 0 and 1.}

\end{frame}

%%%%%%%%%%%%%%%%%%%%%%%%%%%%%%%%%%%%

\begin{frame}
\frametitle{}

Now we will take a sequence of M\&Ms, and you record the number of yellows in the first n draws.

\begin{itemize}
\item $n = 5 \rightarrow$ $\red{R}\green{G}\yellow{Y}\blue{B}\orange{O}$
\item $n = 10 \rightarrow$ $\red{R}\green{G}\yellow{Y}\blue{B}\orange{O} \quad \blue{B}\blue{B}\green{G}\orange{O}\yellow{Y}$
\item $n = 15 \rightarrow$ $\red{R}\green{G}\yellow{Y}\blue{B}\orange{O} \quad \blue{B}\blue{B}\green{G}\orange{O}\yellow{Y} \quad  \yellow{Y}\red{R}\blue{B}\red{R}\red{R} $
\item $n = 20 \rightarrow$ $\red{R}\green{G}\yellow{Y}\blue{B}\orange{O} \quad \blue{B}\blue{B}\green{G}\orange{O}\yellow{Y} \quad  \yellow{Y}\red{R}\blue{B}\red{R}\red{R} \quad  \green{G}\orange{O}\red{R}\blue{B}\yellow{Y}$
\end{itemize}

%\[ \red{R}\green{G}\yellow{Y}\blue{B}\orange{O} \qquad \blue{B}\blue{B}\green{G}\orange{O}\yellow{Y} \qquad  \yellow{Y}\red{R}\blue{B}\red{R}\red{R} \qquad  \green{G}\orange{O}\red{R}\blue{B}\yellow{Y} \]

\app{FR.1 Frequentist inference}{
\begin{enumerate} 

\item What is your sample size? This is your $n$.

\item How many yellows are in your sample? This is your $k$.

\item Calculate the p-value using the Binomial distribution:

p-value = P(k or more yellows $|$ n, \%yellow is 10\%)

\item Do you reject the null hypothesis based on the $\alpha$ you chose earlier?

\item What is the conclusion of your hypothesis test, i.e. what do you report to your boss?

\end{enumerate}
}

\vfill

\end{frame}

%%%%%%%%%%%%%%%%%%%%%%%%%%%%%%%%%%%%

%\begin{frame}[fragile]
%\frametitle{}
%
%Remember $Binomial(n, k) = {n \choose k} p^k (1-p)^{(n-k)}$ \\
%
%$\:$ \\
%\pause 
%
%Say you picked $n = 5$, and hence $k = 1$.
%
%\begin{eqnarray*}
%&&P(1~or~more~yellows~|~n = 5,~\%yellow~is~10\%) \\
%&=& P(K = 1) + P(K = 2) + P(K = 3) + P(K = 4) + P(K = 5) \\
%\pause
%&=& \left[ {5 \choose 1} 0.1^1 \times 0.9^4 \right]+ \left[{5 \choose 2} 0.1^2 \times 0.9^3 \right] + \cdots + \left[{5 \choose 5} 0.1^5 \times 0.9^0 \right] \\
%\pause
%&=& 0.32805 + 0.0729 + 0.0081 +  0.00045 + 0.00001 \\
%\pause
%&\approx& 0.41
%\end{eqnarray*}
%
%
%\end{frame}

%%%%%%%%%%%%%%%%%%%%%%%%%%%%%%%%%%%%

%\begin{frame}[fragile]
%\frametitle{}
%
%Alternatively: \\
%\pause
%{\footnotesize
%\begin{Verbatim}[frame=single, formatcom=\color{blue}]
%# P(K = 1, n = 5, p = 0.1) 
%dbinom(1, 5, 0.1)
%\end{Verbatim}
%}
%{\footnotesize
%\begin{Verbatim}[frame=single, formatcom=\color{gray}]
%0.32805
%\end{Verbatim}
%}
%
%... and
%
%\pause
%{\footnotesize
%\begin{Verbatim}[frame=single, formatcom=\color{blue}]
%# P(K >= 1, n = 5, p = 0.1) 
%sum(dbinom(1:5, 5, 0.1))
%\end{Verbatim}
%}
%{\footnotesize
%\begin{Verbatim}[frame=single, formatcom=\color{gray}]
%0.40951
%\end{Verbatim}
%}

%> round(sum(dbinom(1:5,5,0.1)), 2)
%[1] 0.41
%> round(sum(dbinom(2:10,10,0.1)), 2)
%[1] 0.26
%> round(sum(dbinom(3:15,15,0.1)), 2)
%[1] 0.18
%> round(sum(dbinom(4:20,20,0.1)), 2)
%[1] 0.13

%\end{frame}

%%%%%%%%%%%%%%%%%%%%%%%%%%%%%%%%%%%%

\subsection{Bayesian inference}

%%%%%%%%%%%%%%%%%%%%%%%%%%%%%%%%%%%%

\begin{frame}
\frametitle{Priors}

Now we will start over, with 1:1 odds for the two competing hypotheses. These are our priors:

\begin{itemize}
\item $H_1$: 10\% yellow M\&Ms $\rightarrow$ $P(H_1: p = 0.10) = 0.5$
\item $H_2$: 20\% yellow M\&Ms $\rightarrow$ $P(H_2: p = 0.10) = 0.5$
\end{itemize}

\end{frame}

%%%%%%%%%%%%%%%%%%%%%%%%%%%%%%%%%%%%

\begin{frame}
\frametitle{Calculating posteriors}

\app{FR.2 Bayesian inference}{Using the same data and Bayes' theorem to calculate the probability the percentage of yellow is 10\% and 20\% given the observed data in your sample, i.e. 
\begin{enumerate}
\item $P(p = 0.10~|~data)$
\item $P(p = 0.20~|~data)$
\end{enumerate}
}

\pause

\vfill

Hint:
\begin{adjustwidth}{-1in}{-1in}
{\scriptsize
\begin{eqnarray*}
&&P(p = 0.10~|~data) = \frac{P(data~|~10\% yellow) \times P(10\% yellow)}{P(data)} \\
\pause
&=& \frac{P(data~|~10\% yellow) \times P(10\% yellow)}{P(data~|~10\% yellow) \times P(10\% yellow) + P(data~|~20\% yellow) \times P(20\% yellow)} \\
\pause
&=& \frac{Binom(k~|~n, p = 0.10) \times P(H_1: p = 0.10)}{Binom(k~|~n, p = 0.10) \times P(H_1: p = 0.10) + Binom(k~|~n, p = 0.20) \times P(H_2: p = 0.20)} 
\end{eqnarray*}
}

\end{adjustwidth}

\end{frame}

%%%%%%%%%%%%%%%%%%%%%%%%%%%%%%%%%%%%

\subsection{Comparison}

%%%%%%%%%%%%%%%%%%%%%%%%%%%%%%%%%%%%

\begin{frame}
\frametitle{Bayesian vs. frequentist inference}

\vfill

\begin{adjustwidth}{-1in}{-1in}

\begin{center}
{\scriptsize
\renewcommand\arraystretch{2}
\begin{tabular}{p{1.8cm} || c | c || c | c }
		& \multicolumn{2}{c ||}{Frequentist: p-value} & \multicolumn{2}{c}{Bayesian: Posterior} \\
\hline
{\tiny \# of yellow M\&Ms in} & P(K $\ge$ k $|$ n, 10\% yellow) & Decision & P(10\% yellow $|$ n, k) & P(20\% yellow $|$ n, k) \\
\hline
{\tiny $n = 5: k = 1$} &  &  &  \\
\hline
{\tiny $n = 10: k = 2$}  & & & \\
\hline
{\tiny $n = 15: k = 3$} & & & \\
\hline
{\tiny $n = 20: k = 4$} & & &
\end{tabular}
}
\end{center}

\end{adjustwidth}

\vfill

\end{frame}

%%%%%%%%%%%%%%%%%%%%%%%%%%%%%%%%%%%%

\begin{frame}
\frametitle{Recap}

\begin{itemize}

\item The frequentist approach (using p-values) does not allow us to reject the null hypothesis of 10\% yellow

\item The Bayesian approach yields a higher posterior probability for 20\% yellow

\item The frequentist approach depends on the null hypothesis heavily (we would get different results if we had set $p = 0.20$ as the null hypothesis), but the Bayesian approach allows you to consider an array of hypotheses at once

\item The Bayesian approach also gives you the actual probabilities you want, $P(hypothesis~|~data)$, and brings basic probability into the context of decision making scenarios more naturally than the frequentist p-value

\end{itemize}

\end{frame}

%%%%%%%%%%%%%%%%%%%%%%%%%%%%%%%%%%%%

\end{document}