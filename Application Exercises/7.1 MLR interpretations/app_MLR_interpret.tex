% -*- TeX-engine: xetex; eval: (auto-fill-mode 0); eval: (visual-line-mode 1); -*-
% Compile with XeLaTeX

\documentclass[11pt]{article}
%%%%%%%%%%%%%%%%
% Packages
%%%%%%%%%%%%%%%%

\usepackage[top=1cm,bottom=1cm,left=1.5cm,right= 1.5cm]{geometry}
\usepackage[parfill]{parskip}
\usepackage{graphicx, fontspec, xcolor,multicol, enumitem, setspace}
\DeclareGraphicsRule{.tif}{png}{.png}{`convert #1 `dirname #1`/`basename #1 .tif`.png}

%%%%%%%%%%%%%%%%
% No page number
%%%%%%%%%%%%%%%%

\pagestyle{empty}

%%%%%%%%%%%%%%%%
% User defined colors
%%%%%%%%%%%%%%%%

% Pantone 2015 Spring colors
% http://iwork3.us/2014/09/16/pantone-2015-spring-fashion-report/
% update each semester or year

\xdefinecolor{custom_blue}{rgb}{0, 0.70, 0.79} % scuba blue
\xdefinecolor{custom_darkBlue}{rgb}{0.11, 0.31, 0.54} % classic blue
\xdefinecolor{custom_orange}{rgb}{0.97, 0.57, 0.34} % tangerine
\xdefinecolor{custom_green}{rgb}{0.49, 0.81, 0.71} % lucite green
\xdefinecolor{custom_red}{rgb}{0.58, 0.32, 0.32} % marsala

\xdefinecolor{custom_lightGray}{rgb}{0.78, 0.80, 0.80} % glacier gray
\xdefinecolor{custom_darkGray}{rgb}{0.54, 0.52, 0.53} % titanium

%%%%%%%%%%%%%%%%
% Color text commands
%%%%%%%%%%%%%%%%

%orange
\newcommand{\orange}[1]{\textit{\textcolor{custom_orange}{#1}}}

% yellow
\newcommand{\yellow}[1]{\textit{\textcolor{yellow}{#1}}}

% blue
\newcommand{\blue}[1]{\textit{\textcolor{blue}{#1}}}

% green
\newcommand{\green}[1]{\textit{\textcolor{custom_green}{#1}}}

% red
\newcommand{\red}[1]{\textit{\textcolor{custom_red}{#1}}}

%%%%%%%%%%%%%%%%
% Coloring titles, links, etc.
%%%%%%%%%%%%%%%%

\usepackage{titlesec}
\titleformat{\section}
{\color{custom_blue}\normalfont\Large\bfseries}
{\color{custom_blue}\thesection}{1em}{}
\titleformat{\subsection}
{\color{custom_blue}\normalfont}
{\color{custom_blue}\thesubsection}{1em}{}

\newcommand{\ttl}[1]{ \textsc{{\LARGE \textbf{{\color{custom_blue} #1} } }}}

\newcommand{\tl}[1]{ \textsc{{\large \textbf{{\color{custom_blue} #1} } }}}

\usepackage[colorlinks=false,pdfborder={0 0 0},urlcolor= custom_orange,colorlinks=true,linkcolor= custom_orange, citecolor= custom_orange,backref=true]{hyperref}

%%%%%%%%%%%%%%%%
% Instructions box
%%%%%%%%%%%%%%%%

\newcommand{\inst}[1]{
\colorbox{custom_blue!20!white!50}{\parbox{\textwidth}{
	\vskip10pt
	\leftskip10pt \rightskip10pt
	#1
	\vskip10pt
}}
\vskip10pt
}

%%%%%%%%%%%%%%%%
% Timing
%%%%%%%%%%%%%%%%

% 15-20 minutes

%%%%%%%%%%%%%%%%
% Sakai link for course
%%%%%%%%%%%%%%%%

% UPDATE FOR OWN COURSE
% LINK TO ASSIGNMENTS TOOL IN SAKAI

\newcommand{\Sakai}[1]
{\href{https://sakai.duke.edu/portal/site/ba0d1c18-ba55-473f-9d70-b6a1f9559bbe/page/9870858b-a1a9-481e-8497-8a6ffe9e5be2}{Sakai}}

%%%%%%%%%%%
% App Ex number    %
%%%%%%%%%%%

% DON'T FORGET TO UPDATE

\newcommand{\appno}[1]
{7.1}

%%%%%%%%%%%%%%
% Turn on/off solutions       %
%%%%%%%%%%%%%%

% Off
\newcommand{\soln}[1]{
\vskip5pt
}

%% On
%\newcommand{\soln}[1]{
%\textit{\textcolor{custom_darkGray}{#1}}
%}

%%%%%%%%%%%%%%%%
% Document
%%%%%%%%%%%%%%%%

\begin{document}
\fontspec[Ligatures=TeX]{Helvetica Neue Light}

Dr. \c{C}etinkaya-Rundel \hfill Data Analysis and Statistical Inference \\

\ttl{Application exercise \appno{}: \\
MLR interpretations}

\inst{Submit your responses on \Sakai{}, under the appropriate assignment. Only one submission per team is required. One team will be randomly selected and their responses will be discussed.}

\section*{Income in the US}

The data come from a random sample of 783 observations from the 2012 ACS.

{\small
\begin{enumerate}
\item \texttt{income}: Yearly income (wages and salaries)
\item \texttt{employment}: Employment status, not in labor force, unemployed, or employed
\item \texttt{hrs\_work}: Weekly hours worked
\item \texttt{race}: Race, White, Black, Asian, or other
\item \texttt{age}: Age
\item \texttt{gender}: gender, male or female
\item \texttt{citizens}: Whether respondent is a US citizen or not
\item \texttt{time\_to\_work}: Travel time to work
\item \texttt{lang}: Language spoken at home, English or other
\item \texttt{married}: Whether respondent is married or not
\item \texttt{edu}: Education level, hs or lower, college, or grad
\item \texttt{disability}: Whether respondent is disabled or not
\item \texttt{birth\_qrtr}: Quarter in which respondent is born, jan thru mar, apr thru jun, jul thru sep, or oct thru dec 
\end{enumerate}
}

The following model was fit to predict income:

\begin{center}
{\small
\begin{tabular}{rrrrr}
  \hline
 & Estimate & Std. Error & t value & Pr($>$$|$t$|$) \\ 
  \hline
(Intercept) & -15342.76 & 11716.57 & -1.31 & 0.19 \\ 
  hrs\_work & 1048.96 & 149.25 & 7.03 & 0.00 \\ 
  raceblack & -7998.99 & 6191.83 & -1.29 & 0.20 \\ 
  raceasian & 29909.80 & 9154.92 & 3.27 & 0.00 \\ 
  raceother & -6756.32 & 7240.08 & -0.93 & 0.35 \\ 
  age & 565.07 & 133.77 & 4.22 & 0.00 \\ 
  genderfemale & -17135.05 & 3705.35 & -4.62 & 0.00 \\ 
  citizenyes & -12907.34 & 8231.66 & -1.57 & 0.12 \\ 
  time\_to\_work & 90.04 & 79.83 & 1.13 & 0.26 \\ 
  langother & -10510.44 & 5447.45 & -1.93 & 0.05 \\ 
  marriedyes & 5409.24 & 3900.76 & 1.39 & 0.17 \\ 
  educollege & 15993.85 & 4098.99 & 3.90 & 0.00 \\ 
  edugrad & 59658.52 & 5660.26 & 10.54 & 0.00 \\ 
  disabilityyes & -14142.79 & 6639.40 & -2.13 & 0.03 \\ 
  birth\_qrtrapr thru jun & -2043.42 & 4978.12 & -0.41 & 0.68 \\ 
  birth\_qrtrjul thru sep & 3036.02 & 4853.19 & 0.63 & 0.53 \\ 
  birth\_qrtroct thru dec & 2674.11 & 5038.45 & 0.53 & 0.60 \\ 
   \hline
\end{tabular}
}
\end{center}

\begin{enumerate}

\item Interpret the intercept.

\item Interpret the slope for hrs\_work.

\item Interpret the slope for gender.

\end{enumerate}

\end{document}